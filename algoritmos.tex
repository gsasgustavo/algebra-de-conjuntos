\documentclass[a4paper]{article}

\usepackage[portuguese, ruled, linesnumbered]{algorithm2e}
\usepackage[portuguese]{babel}
\usepackage[utf8x]{inputenc}
\usepackage[T1]{fontenc}
\usepackage[top=3cm, bottom=2cm, left=3cm, rigth=2]{geometry}
\usepackage{arevmath}
\usepackage{graphicx}
\usepackage{algpseudocode}


\begin{document}

  \title{Operações da Álgebra de Conjuntos}
  \author{Ewerton Santiago, Gustavo Siqueira, Wllynilson Carneiro}
  \date{\today}
  \maketitle
  
    \section{Inserção de Elementos}
    \begin{algorithm}[H]
    \label{alg1}
    \caption {\textsc{Inserção($A, a$)}}
    \Entrada {$A$ é um conjunto, $a$ é um valor que será inserido em $A$}
    \Saida{}
    \SetAlgoLined
      \Inicio {
        \Se{$a$ \notin $A$}{
            $N \leftarrow |A|$ \\
            $A_N_+_1 \leftarrow a$
        }
      }
    \end{algorithm}
    
    \section{União}
    \begin{algorithm}[H]
    \caption{\textsc{União}($A$ e $B$)}
    \Entrada{A e B são conjuntos}
    \Saida{O conjunto resultante da União entre os conjuntos A e B}
    \Inicio{
        C \leftarrow \{\}  \\
        \Para{cada $a \epsilon A$}{ 
            Insere($C,a$)
        }
        \Para{cada $b \epsilon B$}{ 
            Insere($C,b$)
        }
        $retorne $C$ $
    }
    \end{algorithm}
    
    
  \section{Pertinência}
  \begin{algorithm}[H]
  \caption {\textsc{Pertinência($a$,$A$)}}
  \Entrada{$a$ é um número e $B$ é um conjunto}
  \Saida{Valor lógico da pertinência do valor $a$ no conjunto $A$}
  \SetAlgoLined
    \Inicio{
        \Se{$a \in A$}{
            retorne $VERDADEIRO$
        }
        \Se{$a \notin A$}{
            retorne $FALSO$
        }
    }
  \end{algorithm}
    
  
  \section{Intersecção}
  \begin{algorithm}[H]
  \caption {\textsc{Intersecção($A$,$B$)}}
  \Entrada{$A$ e $B$ são conjuntos}
  \Saida{O conjunto resultante da intersecção entre os conjuntos $A$ e $B$}
  \SetAlgoLined
    \Inicio{
        $C$ \leftarrow \{\} \\
        \Para{cada $a \in A$}{
            \Se{$a \in B$}{
                $N \leftarrow |C|$ \\
                $C_N_+_1 \leftarrow a$
            }
        }
        retorne $C$
    }
  \end{algorithm}
  
  
  \section{Diferença}
  \begin{algorithm}[H]
  \caption {\textsc{Diferença($A$,$B$)}}
  \Entrada{$A$ e $B$ são conjuntos}
  \Saida{O conjunto resultante da diferença entre os conjuntos $A$ e $B$}
  \SetAlgoLined
    \Inicio{
        $C$ \leftarrow \{\} \\
        \Para{cada $a \in A$}{
            \Se{$a \notin B$}{
                $N \leftarrow |C|$ \\
                $C_N_+_1 \leftarrow a$
            }
        }
        retorne $C$
    }
  \end{algorithm}
  
  
  \section{Produto Cartesiano}
  \begin{algorithm}[H]
  \caption {\textsc{Produto Cartesiano($A$,$B$)}}
  \Entrada{$A$ e $B$ são conjuntos}
  \Saida{O produto cartesiano entre os conjuntos $A$ e $B$}
  \SetAlgoLined
    \Inicio{
        \Para{cada $x \in A$}{
            \Para{cada $y \in B$}{
                imprima($x,y$)
            }
        }
    }
  \end{algorithm}
  
  
  \section{Complemento}
  \begin{algorithm}[H]
  \caption {\textsc{Complemento($A$,$B$)}}
  \Entrada{$A$ e $B$ são conjuntos}
  \Saida{Os conjuntos resultantes do complemento entre os conjuntos $A$ e $B$}
  \SetAlgoLined
    \Inicio{
        $C$ \leftarrow \{\} \\
        $D$ \leftarrow \{\} \\
        \Para{cada $a \in A$}{
            \Se{$a \notin B$}{
                $N \leftarrow |C|$ \\
                $C_N_+_1 \leftarrow a$
            }
        }
        retorne $C$ \\
        \Para{cada $b \in B$}{
            \Se{$b \notin A$}{
                $N \leftarrow |D|$ \\
                $D_N_+_1 \leftarrow b$
            }
        }
        retorne $D$
    }
  \end{algorithm}
  
  
  \section{Conjunto das Partes}
  \begin{algorithm}[H]
  \caption {\textsc{Conjunto das Partes($A$,$B$)}}
  \Entrada{$A$ e $B$ são conjuntos}
  \Saida{O conjunto resultante das partes dos conjuntos $A$ e $B$}
  \SetAlgoLined
    \Inicio{
    }
  \end{algorithm}
  
  
  \section{União Disjunta}
  \begin{algorithm}[H]
  \caption {\textsc{União Disjunta($A$,$B$)}}
  \Entrada{$A$ e $B$ são conjuntos}
  \Saida{O conjunto resultante da união disjunta entre os conjuntos $A$ e $B$}
  \SetAlgoLined
    \Inicio{
    }
  \end{algorithm}
  
  \end{document}
